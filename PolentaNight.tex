\documentclass[12pt]{article}
%%%%%%%%%%%%%%%%%%%%%%%%%%%%%%%%%%%%%%%%%%%%%%%%%%%%%%%%%%%%%%%%%%%%%%%%%%%%%%%%%%%%%%%%%%%%%%%%%%%%%%%%%%%%%%%%%%%%%%%%%%%%%%%%%%%%%%%%%%%%%%%%%%
\usepackage{amsfonts}
\usepackage{graphicx}
\usepackage{amsmath}
\usepackage{amssymb}
\usepackage{float}
\usepackage{booktabs}
\usepackage{multirow}
\usepackage[english]{babel}
\usepackage[skip=2pt]{caption}
\usepackage{tcolorbox}

\setcounter{MaxMatrixCols}{10}

\newtheorem{theorem}{Theorem}
\newtheorem{acknowledgement}{Acknowledgement}
\newtheorem{algorithm}{Algorithm}
\newtheorem{axiom}{Axiom}
\newtheorem{assu}{Assumption}
\newtheorem{case}{Case}
\newtheorem{claim}{Claim}
\newtheorem{conclusion}{Conclusion}
\newtheorem{condition}{Condition}
\newtheorem{conjecture}{Conjecture}
\newtheorem{corollary}{Corollary}
\newtheorem{criterion}{Criterion}
\newtheorem{definition}{Definition}
\newtheorem{example}{Example}
\newtheorem{exercise}{Exercise}
\newtheorem{lemma}{Lemma}
\newtheorem{notation}{Notation}
\newtheorem{problem}{Problem}
\newtheorem{proposition}{Proposition}
\newtheorem{remark}{Remark}
\newtheorem{solution}{Solution}
\newtheorem{summary}{Summary}
\newenvironment{proof}[1][Proof]{\textbf{#1.} }{\ \rule{0.5em}{0.5em}}
\oddsidemargin -30pt
\textwidth 530pt
\topmargin -30pt
\textheight 670pt
\parindent 0in
\parskip 0.3\baselineskip
\renewcommand{\baselinestretch}{1.5} 
\setlength{\footnotesep}{\baselineskip}
\parindent 15pt
\definecolor{uncertainPurple}{RGB}{155,0,255}
\newenvironment{MyColorPar}[1]{\leavevmode\color{#1}\ignorespaces}{}
%\input{tcilatex}
\begin{document}

\author{\setcounter{footnote}{3} Sandro Brusco\thanks{
Department of Economics and College of Business, Stony Brook University}}
\title{The Cultural and Gastronomical Importance of Polenta. An Essay\thanks{
This paper has been written in preparation of the 'Polenta Night' conference to be held in East Setauket on XXX. I would like to thank all the hungry graduate students at the Department of Economics of Stony Brook  University, and especially Camilo Rubbini, for inspiring me to write this very important piece of research. At this point I should put the usual disclaimer about errors and stuff, but I won't. I make many mistakes, but not when I talk about polenta.}}
\date{\today }
\maketitle

\begin{abstract}
Polenta is an under-appreciated food. The main reason is that it is impossible to appreciate polenta enough. This paper traces the origin of polenta, tracks its development as part of the Italian diet and its spreading around the world following Italian immigrants. The cultural and economic importance of polenta is evaluated, as well as its development from basic staple of the poorest part of the Italian society, the landless peasants, to its current status as high-quality food. 4457479

At last this paper provides gastronomic advice on how to use the different qualities of polenta. Polenta can be prepared in many different ways. Since it is so good, any type of polenta can be adapted to any particular dish, but it is true that certain types of polenta are particularly apt for the preparation of certain dishes. To this end, the paper provides practical tips and recipes.
\end{abstract}

\newpage

\section{Introduction}

\label{introduction} Polenta used to be an important part of the Northern Italian diet.

\section{Discussion of the Literature}

\label{literature} The impact of polenta

\section{The Model}

\label{model} Time is discrete and the 

\section{Optimal Contracts under Asymmetric Information}

\label{incomplete} In y.

\subsection{Feasible Contracts}

\label{feasible} A finetting $V_t( h_t) \equiv V_t^{\mathbf{r}}( h_t) $.

\subsection{Efficient Contracts}

In order to analyze thent contract.

\subsection{The Value of Equity and the Value of the Firm}

\label{prelim} The next proposition provides further properties.

\section{The Empirical Distribution of Size}

\label{sizerelevance} In the rest of the paper we w compare the th
\subsection{The Joint Distribution of $V$ and $K$}

When a firm is creat

\begin{minipage}[t]{0.33\textwidth}
	\begin{figure}[H]
	\center
	\includegraphics[width=\textwidth]{Figura1JointPDFVK50periodi}
	\caption{$h_{50}(V,K)$}
	\label{fig:fig1}
\end{figure}
\end{minipage}%
\begin{minipage}[t]{0.33\textwidth}
		\begin{figure}[H]
		\center
	\includegraphics[width=\textwidth]{Figura2JointPDFVK100periodi}
	\caption{$h_{100}(V,K)$}
	\label{fig:fig2}
\end{figure}
\end{minipage}%
\begin{minipage}[t]{0.33\textwidth}
		\begin{figure}[H]
		\center
	\includegraphics[width=\textwidth]{Figura3JointPDFVKmixture}
	\caption{$h^*(V,K)$}
	\label{fig:fig3}
\end{figure}
\end{minipage}

\bigskip

\noindent Figures 

\subsection{The Marginal Distribution of $K$}

While the empirica.

\noindent 
\begin{minipage}[t]{0.5\textwidth} % 1
	\begin{figure}[H]
		\center
		\includegraphics[width=\textwidth]{Figure4MarginalPDFK10periodi}
		\caption{$\gamma _{10}(K)$}
		\label{fig:fig4}
	\end{figure}
\end{minipage}
\begin{minipage}[t]{0.5\textwidth} % 2
	\begin{figure}[H]
		\center
		\includegraphics[width=\textwidth]{Figure5MarginalPDFK50periodi}
		\caption{$\gamma _{50}(K)$}
		\label{fig:fig5}
	\end{figure}
\end{minipage}\newline
\begin{minipage}[t]{0.5\textwidth} % 3
	\begin{figure}[H]
		\center
		\includegraphics[width=\textwidth]{Figure6MarginalPDFK100periodi}
		\caption{$\gamma _{100}(K)$}
		\label{fig:fig6}
	\end{figure}
\end{minipage}
\begin{minipage}[t]{0.5\textwidth} % 4
	\begin{figure}[H]
		\center
		\includegraphics[width=\textwidth]{Figure7MarginalPDFKmixture}
		\caption{$\gamma^*(K)$}
		\label{fig:fig7}
		\end{figure}
	\end{minipage}

\bigskip

\noindent This is 

\section{The Effect of Size on Survival and Growth}

\label{sizereal}In this variables.

\subsection{Size and Survival}

In our model the probabiin $K$ in the sense of first-order stochastic
dominance.

We c
\begin{minipage}[t]{0.33\textwidth}
	\begin{figure}[H]
		\center
		\includegraphics[width=\textwidth]{Figure8CDFofVgivenKatageT=50}
		\caption{$Q_{50} ( \left. V\right\vert	K) $}
		\label{fig:fig8}
	\end{figure}
\end{minipage}
\begin{minipage}[t]{0.33\textwidth}
	\begin{figure}[H]
		\center
		\includegraphics[width=\textwidth]{Figure9CDFofVgivenKatageT=100}
		\caption{$Q_{100}\left( \left. V\right\vert K\right) $}
		\label{fig:fig9}
	\end{figure}
\end{minipage}
\begin{minipage}[t]{0.33\textwidth}
	\begin{figure}[H]
		\center
		\includegraphics[width=\textwidth]{Figure10CDFofVgivenKmixedages}
		\caption{$Q^{\ast}\left( \left. V\right\vert K\right) $}
		\label{fig:fig10}
	\end{figure}
\end{minipage}

\bigskip

\noindent Define 
\subsection{Size and Growth}

Growth opportunit
\section{The Effect of Size on Financial Variables}

start-up firms.

\subsection{Size and Rate of Return}

state is $(V,K) $ as%


\subsection{Size and Capital Structure}

coefficient for cash flow.

\section{Conclusions}

\label{conclusions}Size has always been recognized as an important

\newpage

\begin{center}
\textbf{\Large Appendix I}\bigskip
\end{center}

\bigskip

\noindent \textbf{Proof of Proposition \ref{Prop:sym}}. \ The only
difficulty in
and 
\newpage

\begin{center}
\textbf{\Large Appendix II}\bigskip
\end{center}

The next table shows the parameters values that we have used.

\begin{table}[]
\centering
\begin{tabular}{ccllc}
\textbf{Parameter} &  & \textbf{Description} &  & \textbf{Value} \\[3pt] 
\cmidrule{1-1}\cmidrule{3-3}\cmidrule{5-5} $p$ &  & Probability of good
outcome &  & 0.8 \\[3pt] 
$a$ &  & Curvature of R \ ($R(K) \equiv \frac{K^a}{a} $) &  & 0.5 \\[3pt] 
$q$ &  & Resale Price &  & 0.8 \\[3pt] 
$S_0$ &  & Scrap Value &  & 100 \\[3pt] 
$d$ &  & Depreciation Rate &  & 0.07 \\[3pt] 
$\delta$ &  & Discount Factor &  & 0.98 \\ 
\bottomrule &  &  &  & 
\end{tabular}%
\caption{With this parametrization $K^*=81.5$ and $V^*=722.3$.}
\label{tab:calibration}
\end{table}

\noindent In the first step the value function and the global optimal
discarding all firms that are liquidated before period $%
T,$ we have built:

\begin{itemize}
\item %[leftmargin=-10pt]

\item[i.] the joint distribution of the firms' sizes and values and all
associated distributions conditional on size (Figures 1 -- 10 ); \, and

\item[ii.] the expected growth rates, and standard deviations, conditional
on firm size (Table 1).
\end{itemize}

\begin{figure}[]
\center
\includegraphics[scale=0.7]{lambdakink}
\caption{Two slices (for $K=1$ and $K=K^*$) of the $\protect\lambda$ policy
for a range of relatively small values of $V$}
\label{fig:figpolicylambda}
\end{figure}

\begin{figure}[]
\center
\includegraphics[scale=0.7]{Kprimekink}
\caption{Two slices (for $K=1$ and $K=K^*$) of the $K^{\prime }$ policy for
a range of relatively small values of $V$}
\label{fig:figpolicyKprime}
\end{figure}

\newpage

\begin{thebibliography}{99}
\bibitem{AlbHop} Rui Albuquerque, R. and H. Hopenhayn (2004), `Optimal
Lending Contracts and Firm Dynamics', \emph{Review of Economic Studies}, 
\textbf{71}(2): 285--315.

\bibitem{Sut} Sutton, J. (1997), `Gibrat's Legacy', \emph{Journal of
Economic Literature}, \textbf{35}: 40--59.
\end{thebibliography}

\end{document}
